%%%%%%%%%%%%%%%%%%%%%%%%%%%%%%%%%%%%%%%%%%%%%%%%%%%%%%%%%%%%%%%
%
%     filename  = "Dissertation Index.tex",
%     version   = "Draft 1",
%     date      = "1/16/2013",
%     authors   = "Nicholas P. Nicoletti",
%     copyright = "Nicholas P. Nicoletti",
%     address   = "Department of Political Science,
%                  516 Park Hall,
%                  University at Buffalo,
%                  Buffalo, NY 14260,
%                  USA",
%     telephone = "(585) 752-5167",
%     email     = "npn@buffalo.edu",
%
%%%%%%%%%%%%%%%%%%%%%%%%%%%%%%%%%%%%%%%%%%%%%%%%%%%%%%%%%%%%%%%
%
% Change History:
%
% Draft Version 1.0 - No Changes.
%
%%%%%%%%%%%%%%%%%%%%%%%%%%%%%%%%%%%%%%%%%%%%%%%%%%%%%%%%%%%%%%%
%
% This is a template file to help get you started using the
% psuthesis.cls for theses and dissertations at Penn State
% University. You will, of course, need to put the
% psuthesis.cls file someplace that LaTeX will find it.
%
% We have set up a directory structure that we find to be clean
% and convenient. You can readjust it to suit your tastes. In
% fact, the structure used by our students is even a little
% more involved and commands are defined to point to the
% various directories.
%
% This document has been set up to be typeset using pdflatex.
% About the only thing you will need to change if typesetting
% using latex is the \DeclareGraphicsExtensions command.
%
% The psuthesis document class uses the same options as the
% book class. In addition, it requires that you have the
% ifthen, calc, setspace, and tocloft packages.
%
% The first additional option specifies the degree type. You
% can choose from:
%     Ph.D. using class option <phd>
%     M.S. using class option <ms>
%     M.Eng. using class option <meng>
%     M.A. using class option <ma>
%     B.S. using class option <bs>
%     B.A. using class option <ba>
%     Honors Baccalaureate using the option <honors>
%
% If you specify either ba or bs in addition to honors, it will
% just use the honors option and ignore the ba or bs option.
%
% The second additional option <inlinechaptertoc> determines
% the formatting of the Chapter entries in the Table of
% Contents. The default sets them as two-line entries (try it).
% If you want them as one-line entries, issue the
% inlinechaptertoc option.
%
% The class option ``honors'' should be used for theses
% submitted to the Schreyer Honors College. This option
% changes the formatting on the Title page so that the
% signatures appear on the Title page. Be sure and comment
% out the command \psusigpage when using this option since it
% is not needed and it messes up the vertical spacing on the
% Title page.
%
% The class option ``honorsdepthead'' adds the signature of the
% department head on the Title page for those baccalaureate
% theses that require this.
%
% The class option ``secondthesissupervisor'' should be used
% for baccalaureate honors degrees if you have a second
% Thesis Supervisor.
%
% The vita is only included with the phd option and it is
% placed at the end of the thesis. The permissions page is only
% included with the ms, meng, and ma options.
%%%%%%%%%%%%%%%%%%%%%%%%%%%%%%%%%%%%%%%%%%%%%%%%%%%%%%%%%%%%%%%
% Only one of the following lines should be used at a time.
\documentclass[phd,12pt]{psuthesis}
%\documentclass[draft,phd,inlinechaptertoc]{psuthesis}
%\documentclass[draft,ms]{psuthesis}
%\documentclass[draft,honorsdepthead,honors]{psuthesis}
%\documentclass[draft,honors]{psuthesis}
%\documentclass[draft,secondthesissupervisor,honors]{psuthesis}
%\documentclass[draft,bs]{psuthesis}


%%%%%%%%%%%%%%%%%%%%%%%%%%%%
% Packages we like to use. %
%%%%%%%%%%%%%%%%%%%%%%%%%%%%
\usepackage{amsmath}
\usepackage{amssymb}
\usepackage{amsthm}
\usepackage{exscale}
\usepackage[mathscr]{eucal}
\usepackage{bm}
\usepackage{eqlist} % Makes for a nice list of symbols.
\usepackage[final]{graphicx}
\usepackage[dvipsnames]{color}
\DeclareGraphicsExtensions{.pdf, .jpg, .png}
\usepackage{natbib}
\usepackage{har2nat}
\usepackage{verbatim}
\usepackage{url}
\usepackage{longtable}
\usepackage{mathpazo}
\usepackage{pstricks}
\usepackage{sgamevar}
\usepackage{egameps}
\def\citeapos#1{\citeauthor{#1}'s \citeyear{#1}}
\newenvironment{my_enumerate}
{\begin{enumerate}
  \setlength{\itemsep}{1pt}
  \setlength{\parskip}{0pt}
  \setlength{\parsep}{0pt}}{\end{enumerate}}
\newenvironment{my_itemize}
{\begin{itemize}
  \setlength{\itemsep}{1pt}
  \setlength{\parskip}{0pt}
  \setlength{\parsep}{0pt}}{\end{itemize}}


%%%%%%%%%%%%%%%%%%%%%%%%
% Setting for fncychap %
%%%%%%%%%%%%%%%%%%%%%%%%
% Comment out or remove the next two lines and you will get
% the standard LaTeX chapter titles. We like these A LOT
% better.
\usepackage[Lenny]{fncychap}
\ChTitleVar{\Huge\sffamily\bfseries}


%%%%%%%%%%%%%%%%%%%%%%%%%%%%%%%
% Use of the hyperref package %
%%%%%%%%%%%%%%%%%%%%%%%%%%%%%%%
%
% This is optional and is included only for those students
% who want to use it.
%
% To use the hyperref package, uncomment the following line:
\usepackage[colorlinks=true,urlcolor=purple,citecolor=green,linkcolor=blue]{hyperref}
%
% Note that you should also uncomment the following line:
\renewcommand{\theHchapter}{\thepart.\thechapter}
%
% to work around some problem hyperref has with the fact
% the psuthesis class has unnumbered pages after which page
% counters are reset.


%%%%%%%%%%%%%%%%%%%%%%%%%%%%%%%%%%%%
% SPECIAL SYMBOLS AND NEW COMMANDS %
%%%%%%%%%%%%%%%%%%%%%%%%%%%%%%%%%%%%
% Place user-defined commands below.

\graphicspath{
{Chapter-2/Figures/}
{Chapter-3/Figures/}
{Chapter-4/Figures/}
}

\usepackage{xspace}
\usepackage{mfirstuc}
\usepackage{multirow}
\usepackage{subcaption}

\newcommand{\engine}{Emulation Engine}
\newcommand*{\eg}{e.g.\@\xspace}
\newcommand*{\ie}{i.e.\@\xspace}
\newcommand{\degree}{$^{\circ}$}

\usepackage{setspace}
\doublespacing



%%%%%%%%%%%%%%%%%%%%%%%%%%%%%%%%%%%%%%%%%
% Renewed Float Parameters              %
% (Makes floats fit better on the page) %
%%%%%%%%%%%%%%%%%%%%%%%%%%%%%%%%%%%%%%%%%
\renewcommand{\floatpagefraction}{0.85}
\renewcommand{\topfraction}      {0.85}
\renewcommand{\bottomfraction}   {0.85}
\renewcommand{\textfraction}     {0.15}

% ----------------------------------------------------------- %

%%%%%%%%%%%%%%%%
% FRONT-MATTER %
%%%%%%%%%%%%%%%%
% Title
\title{Designing Multi-Drone Networks and Applications}

% Author and Date of Degree Conferral or Defense
\author{SayedJalil Modares Najafabadi}
% the degree will be conferred on this date
\degreedate{March 2017}
% year of your copyright. I have removed this from the cover page because UB's guidelines do not include it.
\copyrightyear{2017}

% This is the document type. For example, this could also be:
%     Comprehensive Document
%     Thesis Proposal
\documenttype{Disseration}
%The department where you will be submitting the document%
\dept{Department of Electrical Engineering}
% This will generally be The Graduate School, though you can
% put anything in here to suit your needs. This has also been removes from the cover page via the psuthesis.cls document because UB guidelines do not allow for it.
\submittedto{The Graduate School}


%%%%%%%%%%%%%%%%%%
% Signatory Page %
%%%%%%%%%%%%%%%%%%
% You can have up to 7 committee members, i.e., one advisor
% and up to 6 readers.
%
% Begin by specifying the number of readers.
\numberofreaders{3}


% Input reader information below. The optional argument, which
% comes first, goes on the second line before the name.
\advisor[Thesis Advisor, Chair of Committee]
        {Nicholas Mastronarde}
        {Professor of Electrical Engineering}

\readerone[Committee Member]
          {Dimitris Pados}
          {Professor of Electrical Engineering}

\readertwo[Committee Member]
          {Weifeng Su}
          {Professor of Electrical Engineering}

\readerthree[Committee Member]
            {Josep M. Jornet}
            {Professor of Electrical Engineering}

% Makes use of LaTeX's include facility. Add as many chapters
% and appendices as you like.
\includeonly{%
Chapter-1/main,%
Chapter-2/main,%
Chapter-3/main,%
Chapter-4/main,%
Chapter-5/main,%
Appendix-A/Appendix-A,%
Appendix-B/Appendix-B%
}

%%%%%%%%%%%%%%%%%
% THE BEGINNING %
%%%%%%%%%%%%%%%%%
\begin{document}
%%%%%%%%%%%%%%%%%%%%%%%%
% Preliminary Material %
%%%%%%%%%%%%%%%%%%%%%%%%
% This command is needed to properly set up the frontmatter.
\frontmatter

%%%%%%%%%%%%%%%%%%%%%%%%%%%%%%%%%%%%%%%%%%%%%%%%%%%%%%%%%%%%%%
% IMPORTANT
%
% The following commands allow you to include all the
% frontmatter in your thesis. If you don't need one or more of
% these items, you can comment it out. Most of these items are
% actually required by the Grad School -- see the Thesis Guide
% for details regarding what is and what is not required for
% your particular degree.
%%%%%%%%%%%%%%%%%%%%%%%%%%%%%%%%%%%%%%%%%%%%%%%%%%%%%%%%%%%%%%
% !!! DO NOT CHANGE THE SEQUENCE OF THESE ITEMS !!!
%%%%%%%%%%%%%%%%%%%%%%%%%%%%%%%%%%%%%%%%%%%%%%%%%%%%%%%%%%%%%%

% Generates the signature page. This is not bound with your
% thesis.
%\psusigpage

% Generates the title page based on info you have provided
% above.
\psutitlepage

%Generates Copyright Page
\copyrightpage{SupplementaryMaterial/Copyright}

\newpage
% Generates the committee page -- this is bound with your
% thesis. If this is an baccalaureate honors thesis, then
% comment out this line.
% \psucommitteepage

% Generates the Epigraph/Dedication. The first argument should
% point to the file containing your Epigraph/Dedication and
% the second argument should be the title of this page.
\thesisdedication{SupplementaryMaterial/Dedication}{Dedication}

% Generates the Acknowledgments. The argument should point to
% the file containing your Acknowledgments.
\thesisacknowledgments{SupplementaryMaterial/Acknowledgments}

% Generates the Table of Contents
\thesistableofcontents

% Generates the List of Tables
\thesislistoftables

% Generates the List of Figures
\thesislistoffigures

% Generates the List of Symbols. The argument should point to
% the file containing your List of Symbols.
% \thesislistofsymbols{SupplementaryMaterial/ListOfSymbols}

% Generates the abstract. The argument should point to the
% file containing your abstract.
\thesisabstract{SupplementaryMaterial/Abstract}


%%%%%%%%%%%%%%%%%%%%%%%%%%%%%%%%%%%%%%%%%%%%%%%%%%%%%%
% This command is needed to get the main part of the %
% document going.                                    %
%%%%%%%%%%%%%%%%%%%%%%%%%%%%%%%%%%%%%%%%%%%%%%%%%%%%%%
\thesismainmatter

%%%%%%%%%%%%%%%%%%%%%%%%%%%%%%%%%%%%%%%%%%%%%%%%%%
% This is an AMS-LaTeX command to allow breaking %
% of displayed equations across pages. Note the  %
% closing the "}" just before the bibliography.  %
%%%%%%%%%%%%%%%%%%%%%%%%%%%%%%%%%%%%%%%%%%%%%%%%%%
\allowdisplaybreaks{
%
%%%%%%%%%%%%%%%%%%%%%%
% THE ACTUAL CONTENT %
%%%%%%%%%%%%%%%%%%%%%%
% Chapters
\chapter{Introduction}\label{chap:intro}

% \vspace{-5pt}
\section{Motivation}\label{sec:ch1:intro}
Micro air vehicles (MAVs\footnote{We use the terms MAV, unmanned aerial vehicle (UAV), and drone interchangeably.}) are entering our daily life for a variety of applications including surveillance, search-and-rescue~\cite{modares-icra17}, emergency first response, package delivery, environmental monitoring, and precision agriculture.
However, designing multi-drone networks and applications poses numerous inter-disciplinary challenges because the underlying communications and networking problems cannot be explored independently from aero-mechanical, sensing, control, embedded systems, and robotics challenges~\cite{jmod2016poster, jmod2016demo}.

\chapter{UB-ANC Drone: A Flexible Airborne Networking and Communications Testbed}\label{chap:drone}

\vspace{-3pt}
\section{Introduction}\label{sec:ch2:intro}

Networked unmanned aerial vehicles (UAVs) have emerged as an important technology for public safety, commercial, and military applications including search and rescue, disaster relief, precision agriculture, environmental monitoring, and C3ISR (i.e., command and control, communications, intelligence, surveillance and reconnaissance). 
However, designing, implementing, and testing UAV networks poses numerous interdisciplinary challenges because the underlying communications and networking problems cannot be explored independently of aero-mechanical, sensing, control,~~embedded systems,~~and robotics challenges. Indeed, UAV networks are fundamentally {\it cyber-physical systems}~\cite{namuduri2012airborne}.




\chapter{UB-ANC Emulator: An Emulation Framework for Multi-Agent Drone Networks}\label{chap:emul}

% \vspace{-5pt}
\section{Introduction}\label{sec:ch3:intro}




\chapter{UB-ANC Planner: Energy Efficient Coverage Path Planning with Multiple Drones}\label{chap:planner}

\section{Introduction}
\label{sec:ch4:intro}
As noted previously, networked unmanned~~aerial~~vehicles~~(UAVs)~~have emerged as an important technology for public safety, commercial, and military applications including search and rescue, disaster relief, precision agriculture, environmental monitoring, and surveillance. Many of these applications require sophisticated mission planning algorithms to coordinate multiple drones to cover an area efficiently. Such scenarios are complicated by the existence of obstacles, such as buildings, requiring detailed planning for effective operation. Although a lot of work has been done on mission planning, optimal mission planning solutions depend heavily on the specific types of vehicles considered (e.g., ground robots, indoor drones, or outdoor drones), their kinematics, and the specific applications. Prior techniques have been optimized for shortest time to completion or control efficiency. However, a major challenge in the realization of such solutions is the limited energy on each drone. 




\chapter{Conclusion}\label{chap:conclusion}

% \vspace{-5pt}
% \section{Conclusion}\label{sec:concld}
While there are many potential and emerging applications for multi-agent drone networks, deployment and testing of such systems is extremely challenging because it requires experience in networking, software systems, robotics, mission planning, and control among others. To mitigate these challenges, in this dissertation, we presented the University at Buffalo's Airborne Networking and Communications testbed (UB-ANC), which aims to facilitate the design of multi-drone networks and applications. UB-ANC comprises three components: the UB-ANC Drone, the UB-ANC Emulator, and the UB-ANC Planner. The UB-ANC Drone is a flexible open drone platform that provides tools for researchers to test and evaluate different mission planning algorithms and network protocols on actual drones. The UB-ANC Emulator aims to make it easy and convenient to design, implement, test, and debug distributed multi-agent mission planning algorithms in software to ensure correct system operation prior to experimentation in the field on actual UB-ANC drones. Finally, UB-ANC Planner is an energy-efficient coverage path planner, which aims to minimize the maximum energy consumption among drones covering an arbitrary area with obstacles. All projects are open source and available online at 
\begin{center}
{\tt \url{https://github.com/jmodares}}.
\end{center}


%%%%%%%%%%%%%%%%%%%%%%%%%%%%%%%%%%%%%%%%%%%%%%%%%%%%%%%%%%%%%%%
% Appendices
%
% Because of a quirk in LaTeX (see p. 48 of The LaTeX
% Companion, 2e), you cannot use \include along with
% \addtocontents if you want things to appear the proper
% sequence. Since the PSU Grad School requires
%%%%%%%%%%%%%%%%%%%%%%%%%%%%%%%%%%%%%%%%%%%%%%%%%%%%%%%%%%%%%%%
\appendix
\include{Appendix-A/Appendix-A}
\include{Appendix-B/Appendix-B}
%%%%%%%%%%%%%%%%%%%%%%%%%%%%%%%%%%%%%%%%%%%%%%%%%%%%%%%%%%%%%%%
% ESM students need to include a Nontechnical Abstract as the %
% last appendix.                                              %
%%%%%%%%%%%%%%%%%%%%%%%%%%%%%%%%%%%%%%%%%%%%%%%%%%%%%%%%%%%%%%%
% This \include command should point to the file containing
% that abstract.
%\include{nontechnical-abstract}
%%%%%%%%%%%%%%%%%%%%%%%%%%%%%%%%%%%%%%%%%%%
} % End of the \allowdisplaybreak command %
%%%%%%%%%%%%%%%%%%%%%%%%%%%%%%%%%%%%%%%%%%%

%%%%%%%%%%%%%%%%
% BIBLIOGRAPHY %
%%%%%%%%%%%%%%%%
% You can use BibTeX or other bibliography facility for your
% bibliography. LaTeX's standard stuff is shown below. If you
% bibtex, then this section should look something like:
   \begin{singlespace}
   \bibliographystyle{unsrt}
   \phantomsection \addcontentsline{toc}{chapter}{Bibliography}
   \bibliography{ref}
   \end{singlespace}

%\begin{singlespace}
%\begin{thebibliography}{99}
%\addcontentsline{toc}{chapter}{Bibliography}
%\frenchspacing

%\bibitem{Wisdom87} J. Wisdom, ``Rotational Dynamics of Irregularly Shaped Natural Satellites,'' \emph{The Astronomical Journal}, Vol.~94, No.~5, 1987  pp. 1350--1360.

%\bibitem{G&H83} J. Guckenheimer and P. Holmes, \emph{Nonlinear Oscillations, Dynamical Systems, and Bifurcations of Vector Fields}, Springer-Verlag, New York, 1983.

%\end{thebibliography}
%\end{singlespace}

\backmatter

% Vita
% \vita{SupplementaryMaterial/Vita}

\end{document}

